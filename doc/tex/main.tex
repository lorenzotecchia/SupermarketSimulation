\documentclass[a4paper, 12pt]{article}
\usepackage[english]{babel}
\usepackage[utf8]{inputenc}
\usepackage{graphicx}
\usepackage{fancyhdr}

\usepackage{enumitem}
\setitemize{noitemsep,topsep=0pt,parsep=0pt,partopsep=0pt}

\pagestyle{fancy}
\fancyhf{}
\lhead{\textit{Laboratorio di Sistemi Operativi - SupermarketSimulation}}
\rfoot{\thepage}
\renewcommand{\headrulewidth}{0.5pt}
\renewcommand{\footrulewidth}{0.5pt}

\begin{document}

\begin{titlepage}

\begin{center}
   \includegraphics[scale=0.4]{Images/logo-federico-II.pdf} 
\end{center}
   
\thispagestyle{empty}

\center

\textsc{\large Università degli studi di Napoli Federico II}

\vspace{0.5in}

\noindent\makebox[\linewidth]{\rule{\linewidth}{1.2pt}}
\textsc{ \textbf{\large Supermarket Simulation }}
\noindent\makebox[\linewidth]{\rule{\linewidth}{1.2pt}}

\vspace{0.5in}

\begin{minipage}{0.30\textwidth}
    \begin{flushleft}
        Francesco Terrecuso \\
        N86004191
    \end{flushleft}
\end{minipage}
\begin{minipage}{0.30\textwidth}
    \begin{center}
        Lorenzo Tecchia \\
        N86004446
    \end{center}
\end{minipage}
\begin{minipage}{0.35\textwidth}
    \begin{flushright}
        Simone Parente Martone \\
        N86004297
    \end{flushright}
\end{minipage}

\vspace{2in}

\textbf{\large Department of Computer Science} \\

\today

\end{titlepage}

\newpage

\section{Introduzione}
Questo documento presenta un progetto sviluppato per l'esame di Laboratorio di 
Sistemi Operativi, A.A 2023/2024, tenuto dalla Prof. Rossi Alessandra.  \\
Il progetto consiste nella simulazione di un supermercato, modellando le 
interazioni tra quest'ultimo, i clienti e le casse in un contesto multi-threaded. 
Questo file si divide in diverse sezioni che forniscono una panoramica dei 
requisiti, delle scelte architetturali e implementative adottate.
\section{Requisiti identificati}
Il progetto prevede la realizzazione di un architettura client-server che simuli 
un supermercato dotato di $K$ casse e avente un limite di $C$ clienti presenti 
contemporaneamente all'interno. \\
All'inizio della simulazione, $C$ clienti entrano nel supermercato 
simultaneamente, successivamente, ogni volta che escono $E$ clienti, possono 
entrarne altri $E$. \\
Ogni cliente ha un tempo variabile dedicato agli acquisti, una volta scaduto questo
tempo, il cliente si mette in fila per pagare, attendendo il proprio turno.
Ultimato il pagamento, il cliente esce dal supermercato. \\
Ogni cassa è gestita da un cassiere che serve i clienti in base a una politica FIFO,
il tempo di servizio è dato da:
\begin{itemize}[noitemsep,topsep=0pt,parsep=0pt,partopsep=0pt]
    \item una componente costante, specifica per ogni cassiere
    \item una componente variabile, linearmente dipendente dal numero di acquisti
\end{itemize}


\section{Scelte architetturali}
L'architettura del sistema si basa sul modello client-server. \\
Sia il client che il server saranno sviluppati in C. Il server gestisce il
funzionamento del supermercato e delle casse, mentre i client simulano il 
comportamento dei clienti: entrano, fanno acquisti e una volta effettuato il 
pagamento, escono dal supermercato. \\
La gestione del supermercato sarà implementata tramite l'uso di thread multipli
per garantire una simulazione fluida e ottimizzata delle operazioni.
Prevediamo l'utilizzo di:
\begin{itemize}[noitemsep,topsep=0pt,parsep=0pt,partopsep=0pt]
    \item Un thread per la supervisione del supermercato.
    \item Un thread per ogni cassa.
    \item Un thread per ogni cliente all'interno del supermercato.
\end{itemize}

La scelta di utilizzare thread invece di processi è stata presa per evitare di
gravare eccessivamente sul sistema, minimizzando l'uso delle risorse e garantendo
migliori prestazioni. \\
L'accesso esclusivo al supermercato e alle casse è garantito dall'utilizzo di 
mutex, che assicurano la sincronizzazione tra i thread, evitando situazioni di
conflitto.
\section{Funzionalità}

\section{Cronologia lavoro}

\begin{center}
    \begin{tabular}{lll}
        \hline
        \textbf{Data} & \textbf{Autore/i} & \textbf{Descrizione} \\
        \hline
        12/11/2024 & Simone Parente Martone & Adattamento della prima versione \\
        \hline
    \end{tabular}
    \end{center}

\end{document}
